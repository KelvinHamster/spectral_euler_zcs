\documentclass[10pt,a4paper]{article}
\usepackage[utf8]{inputenc}
\usepackage[english]{babel}
\usepackage[english]{isodate}
\usepackage[parfill]{parskip}
\usepackage{amsmath}
\usepackage{mathrsfs}


\DeclareMathOperator*{\argmax}{arg\,max}
\DeclareMathOperator*{\argmin}{arg\,min}

\title{Euler Simulation, Python}
\date{}


\newenvironment{arglist}
    {\begin{center}
    \begin{tabular}{l|p{12cm}}
    argument & description\\
    \hline\\
    }
    { 
    \end{tabular} 
    \end{center}
    }


\newenvironment{optarglist}
    {\begin{center}
    \begin{tabular}{l|p{10cm}|l}
    argument & description & default\\
    \hline\\
    }
    { 
    \end{tabular} 
    \end{center}
    }
\begin{document}
\maketitle

\pagenumbering{roman}
\tableofcontents
\newpage


\pagenumbering{arabic}
\section{Introduction}

The Euler Simulation project applies the methods used in Knowles and Yeh (2018) to simulate, in one dimension, the shoaling process of a shallow-water wave. The original paper uses such a simulation to predict the wave amplification process, that is, how $a$, the amplitude of the wave, relates to $h$, the local water depth.

This project uses the same algorithm, in python, to handle other cases, e.g. the wind's effect on a shoaling solitary wave.
\pagebreak


\section{Code Documentation}

Simulation involves two python files.

\subsection{simulator.py}
\textit{simulator.py} contains one relevant class: \texttt{Simulator1D}, which wraps necessary functionality to run a simulation.





\subsubsection{\texttt{Simulator1D}: Constructor}
\texttt{sim = Simulator1D(bathymetry, dt, dx, eta0, phiS0)}

Initializes a Simulator instance with the given parameters:
\begin{arglist}
            bathymetry  &  numpy array of the bathymetry, must have an even
                            number of nodes. 0 is expected to be water level.\\\hline
            dt          &  time resolution of the simulation\\\hline
            dx          &  spatial resolution of the simulation (distance
                            between points of bathymetry)\\\hline
            eta0        &  initial free surface heights (with spatial
                            resolution dx, 0 is expected for still water)
                            expected to be a numpy array\\\hline
            phiS0       &  velocity potential at the free surface, expected
                            to be a numpy array.
\end{arglist}

        \texttt{Simulator1D} has the following keyword arguments:

\begin{optarglist}
	 zeta\_x & First derivative (gradient) of the bathymetry, if a higher order approximation of
		gradient is desired. If no argument for zeta\_x is passed, it is instead calculated from the finite difference
$$f'(x) = \frac{f(x+h)-f(x-h)}{2h} + O(h^2)$$ where the edges are assumed to have a gradient of 0& (see description)\\\hline
            M    & Terms in the pertubation expansion (higher number is more
                    accurate, but requires more computation)& 5\\\hline
            v    & lowpass threshold. If v is a number, any wavenumber greater
                    than the largest wavenumber times v is clipped off each
                    timestep. v can also be a function that takes a wavenumber magnitude
                    and the peak wavenumber, and returns how it should be scaled
                    & 0.7\\\hline
            g    & acceleration due to gravity. & 9.81\\\hline
            h0   & base still water depth &bathymetry[0]
\end{optarglist}
\pagebreak

In the case that $v$ is a function, the wave number is the first argument and the peak wavenumber is the second argument, so the following will result in the same low-pass filter $v=0.7$:

\begin{verbatim}
def v(k, peak):
    if k/peak <= 0.7:
        return 1
    else:
        return 0
\end{verbatim}





\subsubsection{\texttt{Simulator1D}: step()}
\texttt{sim.step("RK4")}

Steps the simulation forward using the given method. Any arguments
        for the method can be specified as optional arguments or keyword
        arguments.
\begin{arglist}
method &
              The method to use. This can be a string or a function. When
                a string is passed the method in integrator.py of the same
                name is used. See section (2.2) for such methods.
\end{arglist}

Optional arguments specific to a method (e.g. tolerance for implicit methods) can be passed into \texttt{step()}, which will be transferred to the method when called. In particular, \texttt{P\_atmos} is common to all integration methods, and is shown below.

\begin{optarglist}
P\_atmos & Atmospheric pressure at the surface. This can be either a function with arguments \texttt{eta,phiS,eta\_x,phiS\_x,w} or a numpy array of length \texttt{Nx}. See the beginning of section (2.2) for more specifics on the function. & array of zeros
\end{optarglist}





\subsubsection{\texttt{Simulator1D}: calculate\_time\_derivatives()}
\texttt{sim.calculate\_time\_derivatives(eta, phiS, zeta, zeta\_x, zeta\_t, P\_a)}

Returns the tuple \texttt{(eta\_t, phiS\_t)} of time derivatives of $\eta$ and $\Phi^S$ respectively. Takes the following arguments:

\begin{arglist}
eta      & free surface height at the given time step; pass
                    simulator.eta if you want the current time derivative.\\\hline
        phiS     & free surface velocity potential at the given time step;
                    pass simulator.phiS if you want the current time derivative\\\hline
        zeta     & bathymetry ($\zeta$), with 0 corresponding with a depth of -h0\\\hline
        zeta\_x   & spatial derivative of $\zeta$\\\hline
        zeta\_t   & time derivative of $\zeta$\\\hline
        P\_a      & atmospheric pressure at every point, should have the
                    same samples as bathymetry. Expected to be a numpy
                    array or a function. See the beginning of section (2.2) for more specifics on the function.
\end{arglist}




\subsubsection{\texttt{Simulator1D}: volume()}
\texttt{sim.volume()}

Returns the integral
$$\int \eta~dt$$
over the bounds of the simulation, which represents the volume of water in the simulation, offset by a constant that depends only on the bathymetry. This value should be invariant in the simulation, and can give a means of measuring the accuracy of the simulation.




\subsubsection{\texttt{Simulator1D}: energy()}
\texttt{sim.energy()}

Calculates the total energy in the system using the formula
$$\int \eta_t\Phi^S + g\eta^2~dt$$
This value should be invariant in the simulation, and can give a means of measuring the accuracy of the simulation.



\subsubsection{\texttt{Simulator1D}: peak\_location()}
\texttt{sim.peak\_location()}

Returns $$\argmax_{x} \eta(x),$$ the value of $x$ that corresponds with the heighest point of the surface. This value is equivalent to the index of the highest eta, times \texttt{dx}, and provides a means of finding the approximate position of a solitary wave.



\subsubsection{\texttt{Simulator1D}: zeta\_at()}
\texttt{sim.zeta\_at(x)}

Returns $\zeta(x)$, using a linear interpolation scheme for non-discrete points. calling \texttt{zeta\_at(i*dx)} is equivalent to evaluating $\texttt{zeta[i]}$.




\subsubsection{\texttt{Simulator1D}: run\_simulation()}
\texttt{sim.run\_simulation(plot\_dt, data\_dt, directory)}

Runs a simulation, time-stepping until a stop-condition is met, and saving plots and/or data at given intervals of time.
\begin{arglist}
saveplot\_dt & the timestep between saved plots. This number is rounded
                    to the nearest multiple of the simulation dt. If this
                    value does not round to a positive number, plots are not
                    saved. Plots are saved as PNG files with a name
                    corresponding to the order it is saved in. A plot with
                    number 'i' represents the data at time 'i*saveplot\_dt'\\\hline

savedata\_dt &
                   the timestep between saved data. This number is rounded
                    to the nearest multiple of the simulation dt. If this
                    value does not round to a positive number, data is not
                    saved. Data is saved as a json file with name 'dat.json'
                    which is created regardless if data should be saved or not.
                    In the case that data is not saved, only the metadata of
                    the simulation is stored.\\\hline
        
        directory &
                   the directory to save the files to. This can also include
                    a prefix to the file. If \texttt{directory="$\sim$/sim/"}, then plots are
                    saved as ``[number].png" in the $\sim$/sim/ directory. If
                    \texttt{directory="$\sim$/sim"}, then plots are saved as
                    ``sim[number].png" in the home directory. If \texttt{directory=None}, no files
                    are saved.\\\hline
\end{arglist}

\texttt{run\_simulation} has the following optional arguments:

\begin{optarglist}
should\_continue &
                   function that determines if a simulation should stop or
                    not. This takes the simulation as an argument and returns a
                    boolean. The simulation is run until should\_continue
                    returns false. By default, this is the lambda function
                    \texttt{sim: sim.t < 500}, which stops the simulation when a time of 500 is reached & (see description)\\\hline
integrator &
function or string that timesteps the simulation.
                    functions should only take the simulation as an argument
                    and return nothing, modifying the passed simulation.
                    Strings should be the name of a method in Integrator1D,
                    which will be called by the simulation. &
                    ``RK4" \\\hline

save\_eta &
                   Parameters for how eta should be saved when data is saved to json.
                    This should be generated using
                    Simulator1D.data\_save\_params(). If None, then eta is not
                    saved. & None\\\hline
        
        save\_phi &
                   Parameters for how phiS should be saved when data is saved to json.
                    This should be generated using
                    Simulator1D.data\_save\_params(). If None, then phiS is not
                    saved. & None\\\hline

        loop\_callback &
                   this function is called after every simulation step. It
                    should be a void function that takes the arguments
                    \texttt{sim, step, plot, data}, where

                    \texttt{sim} is the simulator at the step

                    \texttt{step} is the integer multiple of dt that the simulation
                    has run

                    \texttt{plot} is a boolean representing if the plot was saved this
                    step

                    \texttt{data} is a boolean representing if the data was saved this
                    step
                    
                    By default, loop\_callback makes a print statement after
                    every 100 time steps. & (see description)\\\hline
        plot\_func &
                   A function that is dedicated to plotting and saving the
                    figure. The function is expected to be void and take the
                    arguments (sim, filename). & (see below)\\\hline
        save\_json &
                   Whether or not to save the file to json. The metadata of
                    the simulation is saved even if savedata\_dt is not
                    positive. Setting save\_json to false prevents this. & False\\\hline
        save\_netcdf&
                   Whether or not to save the file to netcdf. The netCDF file
                    ignores data truncation specifications of save\_eta and
                    save\_phi. & True \\\hline
        save\_buffer
                  & number of datapoints to buffer in between file-writes. If 0
                    or 1, then every savedata\_dt, the json/netCDF file is opened
                    and written to. & 10 \\\hline
        cdf\_h\_invariant
                  & Whether or not h is treated as invariant. If false, then
                    the bathymetry is saved every frame, alongside $\eta$ and
                    $\phi^S$. & True\\\hline
        cdf\_Pderiv
                  &The string that should populate the P\_deriv field in the
                    netcdf file. & ``zero"\\\hline
        cdf\_timeunits
                  & The string that specifies the units of time for the
                    simulation. & ``seconds"\\\hline
        cdf\_spaceunits
                  & The string that specifies the units of x for the
                    simulation. & ``h0*meters"
\end{optarglist}

By default, plot\_func is defined as the function

\begin{verbatim}
def plot_func(sim, filename):
    plt.plot(sim.x, sim.eta, "b")
    plt.plot(sim.x, sim.zeta - sim.h0, "k")
    plt.ylabel("z")
    plt.xlabel("x")
    plt.title(f"dx={sim.dx},dt={sim.dt},t={round(sim.t,3)}")
    plt.savefig(filename)
    plt.clf()
\end{verbatim}

where \texttt{plt} is \texttt{matplotlib.pyplot}.



\subsubsection{\texttt{Simulator1D}: data\_save\_params()}
\texttt{Simulator1D.data\_save\_params()}

Returns a dictionary of parameters for how to save data from a simulation. The output of \texttt{data\_save\_params()} should be used for aguments \texttt{save\_eta} and \texttt{save\_phi} in \texttt{run\_simulation()}.

\begin{optarglist}
        dx    
              & The spatial resolution to save with. If None, then the
                resolution is the same as the simulation. This value
                will always be rounded to a whole number multiple of
                the simulation dx. & None \\\hline

        point\_conversion
              & A boolean that represents if data should be coded as
                a vector (array), or if the vector should be converted
                into a list of (x,y) points. If true, then the conversion
                is made. & False \\\hline

        eps
              & The tolerance of the save data. The data is rounded to the 
                nearest multiple of eps. That is, with eps=0.001, the data
                is saved up to the 3rd decimal place. 0 corresponds with
                no rounding. & 0\\\hline
        
        lin\_tol
             & Only used when point\_conversion is true. Specifies a tolerance
                for which points should not be saved when they are close enough
                to a linear interpolation of the data. If the points are 
                \{(0,0),(0.5,0.5),(1,1)\}, any nonnegative tolerance will discard
                (0.5,0.5). If no points should be discarded, a negative value
                should be given. & -1\\\hline
        
        zero\_trunc
              & If a value is less than this distance from 0, the value is 
                truncated to 0 before saving.& 0
\end{optarglist}




\subsubsection{\texttt{Simulator1D}: soliton()}
\texttt{Simulator1D.soliton(x0,a0,h0,Nx,dx)}

Returns a tuple \texttt{(eta,phiS)} corresponding to the intial conditions of $\eta$ and $\Phi^S$ of a soliton at a given point in space.

\begin{arglist}
        x0   & The x coordinate of the soliton, where
                x=0 corresponds with an index of 0 in the vectorization
                of $\eta$ and $\Phi^S$\\\hline
        a0   & The amplitude of the soliton\\\hline
        h0   & The water deph beneath the soliton\\\hline
        Nx   & The number of points in the vectorization of $\eta$ and $\Phi^S$\\\hline
        dx   & The spatial resolution (distance between points)\\\hline
\end{arglist}

\begin{optarglist}
g & acceleration due to gravity & 9.81\\
\end{optarglist}

This can be used in the constructor of \texttt{Simulator1D} through unpacking:

\texttt{sim = Simulator1D(bathymetry, dt, dx, *Simulator1D.soliton(x0,a0,h0,Nx,dx))}





\subsubsection{\texttt{Simulator1D}: KY\_bathym()}
\texttt{Simulator1D.KY\_bathym()}

Produces a bathymetry profile similar to Knowles and Yeh's paper.
        Expects h0 = 1, but the result can be multiplied by the desired h0 if not 1.

\begin{optarglist}

        Nx    & number of points & $2^{14}$\\\hline
        dx    & spatial resolution (distance between each point) & 0.04\\\hline
        s0    &  nominal slope of the bathymetry & 0.002 \\\hline
        d0    & height of the beach plateau & 0.9 \\\hline
        gamma & smoothing parameter & 0.1 \\\hline
        X1    & position where the bathymetry should start sloping up & 4\\\hline
\end{optarglist}






\subsubsection{\texttt{Simulator1D}: KY\_sim()}
\texttt{sim = Simulator1D.KY\_sim()}

Returns a new simulator similar to Knowles and Yeh's
        initial conditions. The bathymetry is produced by \texttt{KY\_bathym()} and the initial $\eta$ and $\Phi^S$ values are produced by \texttt{soliton()}.

\begin{optarglist}
        Nx       & Number of points (nodes) in the discreteized simulation & $2^{14}$\\\hline
        dx        & Spatial resolution & 0.04 \\\hline
        dt        & Temporal resolution (time step) & 0.01 \\\hline
        s0        & Slope of the bathymetry & 1/500 \\\hline
        x0        & location of the center of the starting soliton & 30 \\\hline
        a0        & amplitude of the soliton & 0.1 \\\hline 
        h0        & depth of the water & 1 \\\hline
\end{optarglist}




\subsubsection{\texttt{Simulator1D}: fields}
In an instance of Simulator1D, the following fields may be of importance:

\begin{arglist}
dt & Temporal resolution (time step) of the simulation. This variable is used by an integrator when stepping the simulation. Most methods will use this time step, but an adaptive method may use a timestep that is smaller. This can be modified externally.\\\hline

dx & Spatial resolution. This is the distance between two points of a vectorized function of $x$. This should not be modified externally.\\\hline

eta & Vectorized $\eta$ with a spatial resolution \texttt{dx} at the current time step. This can be modified externally, but must have the same length (Nx).\\\hline

phiS & Vectorized $\Phi^S$ with a spatial resolution \texttt{dx} at the current time step. This can be modified externally, but must have the same length (Nx). \\\hline

M & Terms in the pertubation expansion of $\Phi$. Calculating the vertical velocity $\Phi_x$ scales approximately $O(M^2)$. This can be modified externally.\\\hline

g & The acceleration due to gravity in this simulation. This can be modified externally.\\\hline

h0 & base still water depth. We approximate the bathymetry as h0 in many calculations. This can be modified externally.\\\hline

zeta & Vectorized bathymetry ($\zeta$), offset so $\zeta = 0$ corresponds with $z = -$h0. This can be modified externally, but must have the same length (Nx). Additionally, zeta\_x should also be changed to the gradient of the new bathymetry.\\\hline

zeta\_x & Vectorized bathymetry gradient ($\nabla\zeta$). This can be modified externally, but must have the same length (Nx). Additionally, zeta should also be changed to match the new bathymetry.\\\hline

Nx & Number of points used in the discretization (vectorization) of the simulation along the $x$-axis. This should not be modified.\\\hline

sim\_length & distance in $x$ that the simulation uses. The vectorizations of $\eta,\Phi^S,$ and $\zeta$ have the domain $[0,$ sim\_length$)$. This should not be modified.\\\hline

x & Vectorized domain. It holds that \texttt{x[i] = dx*i}. This should not be modified.\\\hline

kxdb & The double-domain of wave number. When performing an FFT on a function $f$ with spacing $dx$, the values of the output correspond to the wavenumber by index. If $V$ is such a vectorization of the function $f$, then $$\big(\mathcal F(f)\big)(\text{\texttt{kxdb[i]}}) \approx \text{\texttt{FFT}}(V)\text{\texttt{[i]}}$$
where $\mathcal F$ is the continuous fourier transform. This should not be modified.\\\hline

kappadb & Normalized wavenumber of kxdb. This is equivalent to \texttt{abs(kxdb)}. This should not be modified\\\hline

chi & Vectorized low-pass filter. Any function in the wavenumber domain can apply the filter by pointwise multiplication. This can be modified externally, but kxdb\_im must also be modified accordingly.\\\hline

kxdb\_im & The precomputed value $ik \chi(k)$, which is computed from \texttt{complex(0,1) * kxdb * chi}. This should only be modified when chi is modified.\\\hline

t & The time the simulation has run. This is only ever incremented by an integration method inside a \texttt{step()} call, and can be freely modified and accessed externally.

\end{arglist}


\subsection{integrator.py}
This python file has the class \texttt{Integrator1D} which contains only static members. Each of which is a function that takes a \texttt{Simulator1D} instance, an atmospheric pressure argument \texttt{P\_atmos}, and potential optional arguments.

\texttt{P\_atmos} can either be a numpy array or a function that takes in the arguments \texttt{eta,phiS,eta\_x,phiS\_x,w} and returns a numpy array. The array should have a length equal to the number of nodes \texttt{Nx} used in the simulator, which matches the length of the \texttt{bathymetry} array passed into the constructor. For example, when measuring the wind effect, one may consider
$$P_a = P\frac{d\eta}{dx}$$
for some constant $P$. Such a function can be expressed as
\begin{verbatim}
def P_atmos(eta,phiS,eta_x,phiS_x,w):
    return P * eta_x
\end{verbatim}


\subsubsection{\texttt{Integrator1D}: euler}
Referenced by string \texttt{"euler"}. Makes one derivative calculation per step, using the method:
$$y_{n+1} =  y_{n} + hf'(y_{n})$$




\subsubsection{\texttt{Integrator1D}: RK4}
Referenced by string \texttt{"RK4"}. Makes 4 derivative calculation per step, using the classic 4 step, 4th order, Runge-Kutta method with the Butcher tableau:
\[\arraycolsep=6pt\def\arraystretch{1.2}\begin{array}{c|cccc}
0 & 0 & 0 & 0 & 0\\
\frac12 & \frac12 & 0 & 0 & 0\\
\frac12 & 0 & \frac12 & 0 & 0\\
1 & 0 & 0 & 1 & 0\\
\hline
&\frac16 & \frac13 & \frac13 & \frac16
\end{array}
\]



\subsubsection{\texttt{Integrator1D}: implicit\_midpoint}
Referenced by string \texttt{"implicit\_midpoint"}. Uses the implicit midpoint rule:
$$y_{n+1} = y_n + hf\left(\frac{y_n + y_{n+1}}{2}\right)$$
This equation is solved using fixed point iteration after an initial guess from euler's method.

Takes additional arguments:

\begin{optarglist}
max\_iters & The most iterations used to achieve the desired tolerance, after which, RK4 is defaulted to. & 100\\\hline
tol & The tolerance allowed for the iteration to stop. & $10^{-10}$
\end{optarglist}




\subsubsection{\texttt{Integrator1D}: AM1}
Referenced by string \texttt{"AM1"}. Uses the one step Adams-Moulton method, the implicit trapezoidal rule:
$$y_{n+1} = y_n + h\frac{f(y_n) + f(y_{n+1})}{2}$$
This equation is solved using fixed point iteration after an initial guess from euler's method.

Takes additional arguments:

\begin{optarglist}
max\_iters & The most iterations used to achieve the desired tolerance, after which, RK4 is defaulted to. & 100\\\hline
tol & The tolerance allowed for the iteration to stop. & $10^{-10}$
\end{optarglist}




\subsubsection{\texttt{Integrator1D}: DIRK3}
Referenced by string \texttt{"DIRK3"}. Uses Nørsett's 3 stage, 4th order diagonally implicit Runge-Kutta method withe the Butcher tableau:
\[\arraycolsep=6pt\def\arraystretch{1.8}\begin{array}{c|ccc}
x & x & 0 & 0\\
\frac12 & \frac12-x & x & 0\\
1-x & 2x & 1 - 4x & x\\
\hline
&\frac{1}{6(1-2x)^2} & \frac{3(1-2x)^2-1}{3(1-2x)^2} & \frac1{6(1-2x)^2}
\end{array}
\]
where $x = 1.06858$.
This equation is solved using fixed point iteration for each stage, where each stage has an initial guess of the former stage, with the first stage's initial guess as the derivative at $y_n$. For example, if $k_1$ and $k_2$ are the results of the first and second stage respectively, the second stage solves
$$k_2 = f\left(y_{n} + \left(\frac{1}{2}-x\right)hk_1 + xhk_2\right)$$
with an initial guess $k_2 = k_1$.

Takes additional arguments:

\begin{optarglist}
max\_iters & The most iterations used to achieve the desired tolerance, after which, RK4 is defaulted to. & 100\\\hline
tol & The tolerance allowed for the iteration to stop. & $10^{-10}$
\end{optarglist}





\pagebreak

\section{Examples}

\subsection{\texttt{KY.py}}

This file defines one function, \texttt{sim\_KY}, which runs a simulation very similar to that of the Knowles and Yeh (2018) simulation.
Run using \texttt{python KY.py}. When run in this manner, \texttt{sim\_KY} is called with the default arguments. When imported in python using \texttt{import KY}, the function is not called, but can freely be called by the user. \texttt{sim\_KY} has the following optional arguments:

\begin{optarglist}
    Nx &
           number of points (nodes) in the simulation.& $2^{14}$ \\\hline
    
    a0  &    
           amplitude of the soliton initial condition & 0.1\\\hline

    h0      &  
           base water depth. (depth at zeta = 0)& 1.0\\\hline

    X0    &   
           position (not vector index) of the soliton relative to the center
            of the simulation space. Negative values correspond to the left
            half of the simulation space.& -280 \\\hline

    d0         &  
           height of the beach plateau (on the right side)& 0.9\\\hline

    s0      &     
           nominal slope of the bathymetry.&$\frac{1}{500}$ \\\hline

    dx     &   
           spatial resolution (distance between points/nodes in simulation
            space)& 0.04\\\hline
    
    dt    &  
           Time step of the simulation& 0.01\\\hline
    
    Xt   &    
           Distance from the soliton wave crest to the beach toe. Positive
            values mean that the beach toe is in front of the wave& 50\\\hline

    mass\_err\_crit &
           Largest deviation in mass (calculated from \texttt{volume()}) allowed before the
            simulation is terminated& 0.5\\\hline
            
    energy\_err\_crit&
           Largest deviation in energy (calculated from \texttt{energy()}) allowed before the
            simulation is terminated& 0.5 \\\hline
    
    tmax      &   
           Largest time in simulation allowed before termination.&500 \\\hline

    v         & 
           The low-pass filter. Takes a number to cut off all wavenumbers
            greater than v*max(k), or takes a function that maps k and max(k)
            to how much the corresponding amplitude should be scaled. See
            the constructor of Simulator1D.& 0.7 \\\hline
    
    M           &
           The number of terms in the pertubation expansion to simulate& 5\\\hline

    g              &
           Acceleration due to gravity&9.81 \\\hline

    tsave    & 
           time in between plot saves. Zero or negative corresponds to no
            saving. This value is rounded to a multiple of dt.& 1\\\hline
    
    save\_dir     &  
           the directory/file prefix for the saved plots. By default, this is ``./KY\_dx[dx]\_dt[dt]\_s[s0]\_a[a0]\_plot", where each
bracketed value is replaced by the relevant quantity. & (see description)\\\hline

 gui &
Whether or not to run this program with matplotlib's plotting features. Use \texttt{gui=False} when running with no GUI access. & True
\end{optarglist}


\end{document}